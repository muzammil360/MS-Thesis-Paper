\section{Introduction}
Railroad trespassing is a widely discussed problem in railroad security. From 2006 to 2015, 2717 deaths and 9595 injuries have been reported to be a direct result of trespassing activity in United States\cite{zhang2018automated}. This amounts to 3.37 casualties every day. According to Federal Railroad Administration (FRA), there are around 210,000 railway crossings in USA and around $61\%$ of them are exposed to potential trespassing activity\cite{zhang2018automated}. Each of these sites pose a risk to both trespassers as well as trains and their passengers and cargo. In most cases, collision with a train proves to be fatal for the trespasser. Aside from human costs, these accidents, whether fatal or not, are exceedingly expensive. Property damage, emergency services, safety investigations, insurance, legal and delay costs account for hundreds of thousands up to millions of dollars per accident\cite{goldberg1998train}. 

%A considerable amount of research has been conducted to understand and find solutions to this problem\cite{chadwick2014highway}. 
Although railroad trespassing related accidents have been shown to be the leading cause of fatality\cite{pelletier1997deaths,matzopoulos1998hours,lobb2003evaluation,evans2003accidental}, it remains an under-researched area\cite{lobb2006trespassing}. One simple solution would be to set up a surveillance network of CCTV cameras and employ human analysts to review the video feed on $24 \times 7$
basis. This could be useful for determining potential trespassing locations and times for more efficient resource utilization, i.e., police officers or relevant personnel (such as social workers) could be sent to potential sites only on a need basis. However, one major limitation of this solution is the overwhelming demand this would impose on number of trained human analysts. These analysts have to review tens of hours of CCTV data from hundreds of cameras, making it a tedious and time-consuming process. Manual processing of this ``big data'' is simply infeasible and non-practical. Further, human analysis has additional drawback of subjectivity and unreliability due to dull and mundane nature of the task\cite{norouznezhad2008high}.

Due to above mentioned reasons, bringing automation and artificial intelligence to tackle this trespassing prevention challenge is of vital importance. Trespassing detection indeed serves as the first step towards any future automated AI-based trespassing prevention solution. A reliable automated trespassing detection system would not only provide detection in a timely manner but may also allow us to develop advanced analytics for trespassing patterns over time. For example, analysis over a period of three months may reveal that a group of youth like to play football during a certain evening time. Certain locations might see increased trespassing during the morning or evening times with people returning home from jobs taking a short-cut. Other locations such as underpasses and bridges may provide a preferred meeting location for drug addicts. Use of ``big data analytics'' can help us to ultimately make better predictions and thus assist in reducing trespassing activity substantially.
\subsection{Goals of this research}
%\\ \textbf{Goals} \\
\label{sec:goal}
Given a surveillance video, the problem of trespassing detection is to decide whether each given frame has human trespassing activity or not. We define a trespasser as a human within the camera field of view. We not only want to predict the occurrence of trespassers but we also want to do so in a time and resource efficient manner. We notice that a railroad surveillance video is extremely sparse in terms of trespassing activity, i.e., in a given 24 hours of railroad surveillance video, most of the video shows no trespassing activity. To this work, we propose to leverage this property of sparseness to reduce the processing time. 
Further, we postulate that the detection performance\footnote{Two terms, ``accuracy'' and ``performance'', have been interchangeably used in text. However, experimental evaluation uses $f1$ score as concrete evaluation metric.} and speed of detection are two conflicting goals. Generally, if one wishes to improve the speed of detection, they will have to sacrifice accuracy and vice versa. Therefore, we are interested in developing a  flexible solution that is capable of trading-off performance with computational time.

\subsection{State of the art}
Despite the obvious need, there has been limited activity among the research community towards solving this problem. Recently, Zhang et. al. \cite{zhang2018automated} proposed a system focusing on detecting the near-misses. However, their proposed methodology makes an important assumption that the train will always constitute the majority of the moving pixels. This strategy may fail if the camera is located away from the train. Additionally, Salmane et. al. \cite{salmane2015video} proposed a system for detecting the hazard situations at railway crossings. However, their system only suffices to detect the moving objects and does not discriminate them into train, vehicle or person as would be required for trespassing detection. Another weakness common to these two methods is that they do not explore advanced CNN based deep learning methods that have been shown to outperform conventional methods \cite{krizhevsky2012imagenet}. 

\subsection{Approach}
%\\ \textbf{Approach}\\ 
Our framework \textbf{ARTS} (\underline{A}utomated \underline{R}ailroad \underline{T}respassing detection \underline{S}ystem) solves the identified problem of automated trespassing detection with envisioned goals by adopting a two-step approach. The first stage is responsible for filtering out frames that show little to no activity, this way reducing the amount of data to be processed by the later extremely compute-intensive stage. The second stage adopts state-of-the-art deep learning model based on CNN to ensure effective detection of trespassing activity.
\subsection{Contributions}
%\\ \textbf{Contributions} \\
Following are the key contributions of this work:
\begin{itemize}
\item We propose a flexible trespassing detection framework called \textbf{ARTS} that can trade-off speed and accuracy while simultaneously leveraging the property of activity sparseness in a dataset. (Sec. \ref{sec:time-accuracy-trafe-off})

\item Our proposed ARTS framework adopts a \textit{plug and play} design to allow embedding any algorithm suitable to individual stages of our framework. (Sec. \ref{sec:proposed-framework})

\item Our solution combines an inexpensive yet effective traditional computer vision approach with a state-of-the-art deep learning architecture to develop an overall robust technology. (Sec. \ref{sec:stage1} and \ref{sec:stage2})

\item We conduct an in-depth experimental evaluation of the proposed \textbf{ARTS} approach to demonstrate the effectiveness.\footnote{The code shall be released to the research community after publication.} (Sec. \ref{sec:exp-eval})



\end{itemize}
\subsection{Organization of paper}
%\textbf{Organization of paper} \\
The remainder of this paper is organized as follows.  Section \ref{sec:related-work} discusses the related work. Section \ref{sec:methodology} explains the technical details of proposed approach while Section \ref{sec:exp-eval} discusses the experimental evaluation. Section \ref{sec:conclusion} concludes the paper and discusses future directions. 