\section{Introduction}
Railroad trespassing is a widely discussed issue in railroad security. From 2006 to 2015, 2717 deaths and 9595 injuries have been reported to be a direct result of trespassing activity in United States\cite{zhang2018automated}. This amounts to 3.37 casualties every day. According to Federal Railroad Administration (FRA), there are around 210,000 railway crossings in USA and around $61\%$ of them are exposed to potential trespassing activity\cite{zhang2018automated}. Each of this site poses a risk to both trespassers as well as train. In most cases, collision with a train proves to be fatal for the trespasser. Aside from human costs, these accidents, whether fatal or not, are exceedingly expensive. Property damage, emergency services, safety investigations, insurance, legal and delay costs may account for hundreds of thousands up to millions of dollars per accident\cite{goldberg1998train}. 

%A considerable amount of research has been conducted to understand and find solutions to this problem\cite{chadwick2014highway}. 
Although, railroad trespassing related accidents have been shown to be the leading cause of fatality\cite{pelletier1997deaths,matzopoulos1998hours,lobb2003evaluation,evans2003accidental}, yet it is still an under-researched area\cite{lobb2006trespassing}. One simple solution is to set up a surveillance network of CCTV cameras and employ human analysts to review the video feed on $24 \times 7$
basis. This can be useful to determine potential trespassing location and time for more efficient resource utilization i.e. police officers or relevant personnel (such as social workers) can be sent to potential sites only as per need basis. However, one major limitation of this solution is the sheer overwhelming requirement of large number of trained human analysts. These analysts have to review tens of hours of CCTV data from hundreds of cameras, making it a tedious and time-consuming process. Manual processing of this ``big data'' is simply infeasible and non-practical. Further, human analysis has additional drawback of subjectivity and unreliability due to dull and mundane nature of task\cite{norouznezhad2008high}.

Due to above mentioned reasons, bringing automation and artificial intelligence to any trespassing prevention solution is of vital importance. Trespassing detection indeed serves as the first step towards any automated AI-based trespassing prevention solution. A reliable automated trespassing detection system will not only provide detection in a timely manner but will also allow us to develop advanced analytics by studying trespassing patterns over time. For example, analysis over a period of three months may reveal that a group of children like to play football during the evening time. Certain locations might see increased trespassing during the morning and/or evening times because people returning home from jobs may want to take a short-cut. Other locations such as underpasses and bridges may provide a preferred meeting location for drug addicts. Use of these advanced ``big data analytics'' can help us make better predictions and thus assist in  reducing trespassing activity substantially.
\subsection{Goals}
%\\ \textbf{Goals} \\
\label{sec:goal}
Given a surveillance video, the problem of trespassing detection is to decide whether each frame has human trespassing activity or not. We define trespasser as a human within the camera field of view. We not only want to predict the label but also want to do so in a time-efficient manner. We notice that railroad surveillance video is sparse in terms of trespassing activity i.e. in a given 24 hours of railroad surveillance video, most of the video shows no trespassing activity. We aim to leverage this property of sparseness to reduce the processing time. 
Further, we postulate that the detection performance and speed (of detection) are two opposite goals. Generally, if one wishes to improve the speed, they will have to sacrifice accuracy and vice versa. Therefore, we are interested in developing a  flexible solution that is capable of trading-off performance with computational time.
\subsection{Approach}
%\\ \textbf{Approach}\\ 
In order to solve the problem with envisioned goals, we take a two-step approach. The first stage is responsible for filtering out frames that show little to no activity, this way reducing the amount of data to be processed by the later more compute-intensive stage which adopts state-of-the-art Faster-RCNN to ensure effective classification of trespassing activity.
\subsection{Contributions}
%\\ \textbf{Contributions} \\
Following are the key contributions of this work. 
\begin{itemize}
\item We propose a flexible trespassing detection framework that can trade-off speed and accuracy while simultaneously leveraging the property of activity sparseness in a dataset. (Sec. \ref{sec:proposed-framework})

\item Our practical system combines an inexpensive yet effective traditional computer vision approach with state of the art deep learning architecture to develop a robust system. (Sec. \ref{sec:stage1} and \ref{sec:stage2})

\item We conduct thorough and in-depth experimental evaluation of the proposed approach to demonstrate the effectiveness.\footnote{The code shall be released to the research community after publication.} (Sec. \ref{sec:exp-eval})
\end{itemize}
\subsection{Organization of paper}
%\textbf{Organization of paper} \\
The remainder of this paper is organized as follows.  Section \ref{sec:related-work} discusses the related work. Section \ref{sec:methodology} explains the technical details of proposed system and Section \ref{sec:exp-eval} discusses the experimental evaluation of proposed approach. Section \ref{sec:conclusion} concludes the paper and discusses possible future directions. 