\section{Introduction}
Railroad trespassing is a widely discussed problem in railroad security. From 2006 to 2015, 2717 deaths and 9595 injuries have been reported to be a direct result of trespassing activity in United States\cite{zhang2018automated}. This amounts to 3.37 casualties every day. According to Federal Railroad Administration (FRA), there are around 210,000 railway crossings in USA and around $61\%$ of them are exposed to potential trespassing activity\cite{zhang2018automated}. Each of these sites pose a risk to both trespassers as well as trains and their passengers and cargo. In most cases, collision with a train proves to be fatal for the trespasser. Aside from these tragic human costs, these accidents, whether fatal or not, are exceedingly expensive. Property damage, emergency services, safety investigations, insurance, legal and delay costs account for hundreds of thousands to millions of dollars per accident\cite{goldberg1998train}. 

%A considerable amount of research has been conducted to understand and find solutions to this problem\cite{chadwick2014highway}. 
Although railroad trespassing related accidents have been shown to be the leading cause of fatality\cite{pelletier1997deaths,matzopoulos1998hours,lobb2003evaluation,evans2003accidental}, it remains an under-researched area\cite{lobb2006trespassing}. One simple solution would be to set up a surveillance network of CCTV cameras and employ human analysts to review the video feed on a $24 \times 7$
basis. This could be useful for determining locations and times for dangerous trespassing incidents with the ultimate objective to develop  more efficient resource utilization i.e. deployment of limited personnel (from police officers to social workers) to potential trespassing sites only on a need basis. However, a major limitation of this solution is the overwhelming demand this would impose on trained human analysts. These analysts have to review tens of hours of CCTV data from hundreds of cameras, making it a tedious and time-consuming process. Manual processing of this ``big data'' is simply infeasible and non-practical. Further, human analysis has additional drawback of subjectivity and unreliability due to the dull and mundane nature of the task\cite{norouznezhad2008high}.

Due to the above mentioned reasons, bringing automation and artificial intelligence to tackle this trespassing prevention challenge is of vital importance. Trespassing detection serves as the first step towards any future automated AI-based trespassing prevention solution. A reliable automated trespassing detection system would not only provide detection in a timely manner but may also allow us to develop advanced analytics for trespassing patterns over time. For example, analysis over a period of three months may reveal that a group of youth likes to play football during a certain evening time near a section of the track. Certain locations might see increased trespassing during the morning or evening times with people taking a short-cut when returning home from jobs. Other locations such as underpasses and bridges may provide a preferred meeting location for drug addicts. Use of ``big data analytics'' can help us to ultimately make better predictions and thus assist in reducing trespassing activity substantially.
\subsection{Goals of this research}
%\\ \textbf{Goals} \\
\label{sec:goal}
Given a surveillance video, the problem of trespassing detection is to decide whether each given frame has human trespassing activity or not. We define a trespasser as a human within the camera field of view. We not only want to predict the occurrence of trespassers but we also want to do so in a time and resource efficient manner. We observe that a railroad surveillance video is extremely sparse in terms of trespassing activity, i.e., in a given 24 hours of railroad surveillance video, most of the video shows no trespassing activity. We thus propose to leverage this property of sparseness to reduce the processing time. 
Further, we postulate that the detection accuracy\footnote{Experimental evaluation uses $f1$ score and AUC as concrete evaluation metrics.} and speed of detection are two conflicting goals. Generally, if one wishes to improve the speed of detection, they will have to sacrifice accuracy and vice versa. Therefore, we are interested in developing a  flexible solution that is capable of trading-off accuracy with computational time.

\subsection{State-of-the-art}
Despite the obvious need, there has been limited activity among the research community towards solving this problem. Recently, Zhang et. al. \cite{zhang2018automated} proposed a technique focusing on detecting the near-misses during railroad trespassing. However, their proposed methodology makes the assumption that the train always constitutes the majority of the moving pixels. Their strategy may thus fail if the camera is located away from the train. Salmane et. al. \cite{salmane2015video} proposed a technique for detecting hazard situations at railway crossings. However, their technique only detects the movement and does not discriminate moving objects into train, vehicle or person as would be required for trespassing detection. Another shortcoming common to these prior two methods is that they do not leverage advanced deep learning methods. Recent research in computer vision has shown that Convolutional Neural Network (CNN) based deep neural network architectures are the model of choice for image and video analysis\cite{krizhevsky2012imagenet}. 

\subsection{Approach}
%\\ \textbf{Approach}\\ 
Our framework \textbf{ARTS} (\underline{A}utomated \underline{R}ailroad \underline{T}respassing detection \underline{S}ystem) solves this problem of automated trespassing detection by adopting a two-step approach. The first stage is responsible for filtering out frames that show little to no activity, this way reducing the amount of data to be processed by the later extremely compute-intensive stage. The second stage adopts a state-of-the-art deep learning model based on CNN to ensure effective detection of trespassing activity. To realize the impact of this approach, consider a scenario in which we have 1 hour of surveillance video (at 10 frames per second). If we were to use a single staged state-of-the-art solution that detects trespassing at every frame, we would need to spend approximately 5 hours. Whereas our proposed ARTS approach only takes 0.8 hours (48 minutes) even when a rather high ratio of activity\footnote{Activity ratio is precisely defined in Sec. \ref{sec:synthetic-dataset-generator}} ($10\%$) were to be present in the above said surveillance video.

\subsection{Contributions}
%\\ \textbf{Contributions} \\
Next we enumerate the key contributions of this work:
\begin{itemize}
\item We propose a flexible trespassing detection framework called \textbf{ARTS} that can trade-off speed and accuracy by leveraging the observed property of activity sparseness. (Sec. \ref{sec:time-accuracy-trafe-off})

\item Our proposed ARTS framework adopts a \textit{plug and play} design to allow embedding any algorithm suitable to individual stages of our framework. (Sec. \ref{sec:proposed-framework})

\item Our solution combines an inexpensive yet effective traditional computer vision approach with a state-of-the-art deep learning architecture to develop an overall robust technology. (Sec. \ref{sec:stage1} and \ref{sec:stage2})

\item We conduct an in-depth experimental evaluation of the proposed \textbf{ARTS} approach considering a variety of experiments to demonstrate relative trade-off and the overall effectiveness of ARTS.\footnote{The code shall be released to the research community after publication.} (Sec. \ref{sec:exp-eval})



\end{itemize}
\subsection{Organization of paper}
%\textbf{Organization of paper} \\
The remainder of this paper is organized as follows.  Section \ref{sec:related-work} discusses the related work. Section \ref{sec:methodology} explains the technical details of proposed approach while Section \ref{sec:exp-eval} discusses the experimental evaluation. Section \ref{sec:conclusion} concludes the paper and discusses future directions. 