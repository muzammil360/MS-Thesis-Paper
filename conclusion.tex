\section{Conclusion and future work}
\label{sec:conclusion}
In this work, we propose and then comprehensively study a flexible trespassing detection solution framework (called ARTS) adopting state-of-the-art deep learning methods. The proposed framework can trade-off processing speed and accuracy. Although initially envisioned for railroad security, the proposed approach has potential applications in video surveillance domains characterized by a sparsity in activity.  The ARTS framework features two stages with the first stage designed to efficiently remove the background frames from the activity frames. The second stage is responsible for differentiating between human trespassing and any other unknown activity. Our proposed ARTS framework adapts a plug and play infrastructure to allow researchers to plug in other algorithms relevant to stage 1 or to stage 2, with ease in the future. The effectiveness of our approach has been demonstrated on a public domain surveillance dataset. 

Future directions include building a trespassing prediction system that uses the output of the ARTS to predict future trespassing events. Another direction is to improve the accuracy of the proposed techniques. We note that the current accuracy is limited by the accuracy of stage 2. Currently stage 2 does not use any temporal information, i.e., each frame is treated independently and thus not conditioned on the previous frames (history). The utilization of the temporal information has the potential of significantly improving the accuracy specially for challenging cases of occlusion and background.