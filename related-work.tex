\section{Related Work}
\label{sec:related-work}
% Since our work is at the cross section of railroad trespassing related safety and use of computer vision in trespassing detection, therefore we shall review the relevant work related to both. 
% Since, our work revolves around using computer vision for railroad trespassing detection, we shall review the work specifically related to it. Additionally, we also review the work related to deep learning based objected detection.  

\begin{comment}
\subsection{Railroad trespassing related safety} Considerable efforts have been made to understand the risks associated with trespassing activity and developing strategies to reduce those risks. Caird et al.\cite{caird2002human} developed a taxonomy of human factors that contribute towards the unsafe human activity.  The taxonomy groups  common accident contributors in 6 categories: unsafe actions, individual differences, train visibility, passive signs, active warning systems and physical constraints. As opposed to Caird et al.\cite{caird2002human}, Sussman and Raslear\cite{sussman2007railroad} indicated intentional, distraction-caused or other (visibility issues or driver confusion) reasons for accidents. All of these issues require different approaches to solve the problem. However, those approaches can broadly be classified as engineering, education and enforcement-based.

Apart from studying the strategies to reduce the risk, efforts have been made to quantify the frequency and severity of accidents. US Department of Transportation (DOT) accident prediction model is the most widely used method to predict expected no. of collisions per year, at a given crossing site \cite{austin2002alternative,tustin1986railroad}. It is based on different parameters related to traffic volume, train time table and previous accident history \cite{chadwick2014highway}. This method has been compared to Transport Canada Accident Model \cite{saccomanno2003identifying} by Chaudhary et al.\cite{chaudhary2011railroad}. Chaudhary et al. reports that overall US DOT model predicts the yearly number of accidents more accurately. However, in cases where the crossing had an accident history, the Transport Canada model outperformed US DOT model. As a conclusion, they suggested to adapt the Transport Canada model to US crossing data and use it to rank the most dangerous crossings.

Different engineering strategies used to prevent collision can be broadly classified as sealed corridors, obstacle detection and traffic channelization\cite{chadwick2014highway}. The concept of sealed corridors presents an ideal solution to the problem, however, high costs and reduced access renders it less attractive. Obstacle detection refers to detecting the presence of person or vehicle on tracks and communicating it to the approaching train\cite{glover09}. It provides a cost-effective and feasible way to reduce collision risk, however the main challenge is the short reaction time available to bring the train to stop. Glover\cite{glover09} suggests that there may be limited reduction in severity of a collision because train may still collide. However, Hall\cite{hall2007reducing} argues that obstacle detection may still be beneficial as it might give invaluable time to decelerate the train sufficiently to save human life. Traffic channelization is another effective strategy which separates the traffic flow from rail tracks. Federal Railroad Administration research shows positive results and suggests that it discourages risky driving behaviour around the crossing\cite{horton2012use}. 
\end{comment}



\subsection{Computer vision in trespassing detection} 
Little prior work has been done to use computer vision for railroad trespassing safety. Shah et al.\cite{shah2007automated} employed color-based background subtraction to detect moving objects. They use color, motion and size-based features to track those objects. Tracked objects are classified into people, a group of people or a vehicle. Salmane et al.\cite{salmane2015video} proposed a multi-stage system that uses frame-based background subtraction for moving object detection. However, moving objects are not discriminated into train, vehicle or persons as would be required for trespassing detection. Further, as noticed by Zhang et al.\cite{zhang2018automated}, even though the system is expected to run in real time, no information regarding the speed of the algorithm is reported. Zhang et al.\cite{zhang2018automated} developed a near-miss trespassing detection system. Their focus is on the detection of near-miss events on gated crossings rather than general trespassing detection. They first determine the time interval during which the gates are closed and then detect moving objects using background subtraction. However, they use the number of pixels as classification metric to distinguish between the train and other objects (people and vehicles). This strategy works for their test data but may fail if the camera is located away from train, since it is based on the simple assumption that the train will always constitute the majority of the moving pixels. 

One drawback common to all the above mentioned systems is that they use simple background subtraction method (based on subtracting mean or median pixel values) for detecting moving objects. This method is known to generate noisy output\cite{stauffer1999adaptive}. As one exception, Shah et al.\cite{shah2007automated} use hand-crafted features to classify the object type. However, recent research suggests that deep-learning based approaches outperform these hand-crafted feature based approaches\cite{benenson2014ten}. This limitation in previous works motivates us to explore the design of a deep learning based trespassing detection solution that can reliably detect human trespassing activity. 

\subsection{Deep learning based object detection}
Since the success of Alexnet\cite{krizhevsky2012imagenet} in the 2011 ILSVRC\cite{russakovsky2015imagenet} challenge, a considerable effort has been put in by the computer vision community to explore deep learning methods. Overfeat\cite{sermanet2013overfeat} was among the first attempts in improving object detection performance using deep learning. However, it was extremely slow as it considered all sliding window positions. Instead of using sliding windows, Girshick et al.\cite{girshick2014rich} proposed to use a RoI (Region of Interest) based approach. They use selective search\cite{uijlings2013selective} to generate RoIs. These RoIs (which are far less in number than sliding windows) are then passed through the same pipeline as Overfeat.  Though, it resulted in significant reduction of time; this solution still applied the expensive convolution operation on each RoI independently. Most of the RoIs are overlapping and thus computational resources are wasted during re-computation of overlapping RoIs. Fast-RCNN\cite{ref_fastrcnn} circumvents this issue by sharing the convolutional features. This technique makes Fast-RCNN 9x faster than RCNN\cite{girshick2014rich} in training and 213x faster in inference\cite{ref_fastrcnn}. Sharing of features speeds up the computation to such a degree that it renders selective search \cite{uijlings2013selective}, a non-deep learning process, the bottle neck in the object detection pipeline. In response, Ren et. al. proposed Faster-RCNN\cite{ref_fasterrcnn} which replaced the conventional selective search process with RPN (Region Proposal Network). RPN, being part of the neural network architecture, is fast as compared to selective search. Both Fast-RCNN and Faster-RCNN are considered to be two-step object detectors in that they propose regions in first step and then proceed to classify them in second step.

As opposed to two-step detectors, single step detectors such as YOLO\cite{redmon2016you,redmon2018yolov3} and SSD\cite{liu2016ssd} have also been recently proposed. These solutions merge the region proposal process and classification process into a single unified deep pipeline leading to faster response times. However, in terms of accuracy, they lack behind two-step detectors since the number of region proposals considered by single-step detectors are far less than those considered by two-step detectors. 










