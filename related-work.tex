\section{Related Work}
Since our work is at the cross section of railroad trespassing related safety and use of computer vision in trespassing detection, therefore we shall review the relevant work related to both. 

\subsection{Railroad trespassing related safety} Considerable efforts have been made to understand the risks associated with trespassing activity and developing strategies to reduce those risks. Caird et al.\cite{caird2002human} developed a taxonomy of human factors that contribute towards the unsafe human activity.  The taxonomy groups  common accident contributors in 6 categories: unsafe actions, individual differences, train visibility, passive signs, active warning systems and physical constraints. As opposed to Caird et al.\cite{caird2002human}, Sussman and Raslear\cite{sussman2007railroad} indicated intentional, distraction-caused or other (visibility issues or driver confusion) reasons for accidents. All of these issues require different approaches to solve the problem. However, those approaches can broadly be classified as engineering, education and enforcement-based.

Apart from studying the strategies to reduce the risk, efforts have been made to quantify the frequency and severity of accidents. US Department of Transportation (DOT) accident prediction model is the most widely used method to predict expected no. of collisions per year, at a given crossing site \cite{austin2002alternative,tustin1986railroad}. It is based on different parameters related to traffic volume, train time table and previous accident history \cite{chadwick2014highway}. This method has been compared to Transport Canada Accident Model \cite{saccomanno2003identifying} by Chaudhary et al.\cite{chaudhary2011railroad}. Chaudhary et al. reports that overall US DOT model predicts the yearly number of accidents more accurately. However, in cases where the crossing had an accident history, the Transport Canada model outperformed US DOT model. As a conclusion, they suggested to adapt the Transport Canada model to US crossing data and use it to rank the most dangerous crossings.

Different engineering strategies used to prevent collision can be broadly classified as sealed corridors, obstacle detection and traffic channelization\cite{chadwick2014highway}. The concept of sealed corridors presents an ideal solution to the problem, however, high costs and reduced access renders it less attractive. Obstacle detection refers to detecting the presence of person or vehicle on tracks and communicating it to the approaching train\cite{glover09}. It provides a cost-effective and feasible way to reduce collision risk, however the main challenge is the short reaction time available to bring the train to stop. Glover\cite{glover09} suggests that there may be limited reduction in severity of a collision because train may still collide. However, Hall\cite{hall2007reducing} argues that obstacle detection may still be beneficial as it might give invaluable time to decelerate the train sufficiently to save human life. Traffic channelization is another effective strategy which separates the traffic flow from rail tracks. Federal Railroad Administration research shows positive results and suggests that it discourages risky driving behaviour around the crossing\cite{horton2012use}. 


%\begin{enumerate}
%    \item understand the risks
%    \item study the accidents (statistics)
%    \item why do them happen (human behaviour)
%    \item ways to prevent/reduce it
%\end{enumerate}

\subsection{Artificial intelligence} Very little prior work has been done to use computer vision for railroad trespassing safety. Shah et. al\cite{shah2007automated} employed the gradient and color-based background subtraction to detect moving objects. They use color, motion and size-based features to track those objects. Tracked objects are classified into people, group of people or vehicle. Salmane et al.\cite{salmane2015video} proposed a multi-stage system that uses frame-based background subtraction for moving object detection in the first stage. However, the moving objects are not discriminated into train, vehicle or persons if the system is to be used for trespassing detection. Further, as noticed by Zhang et al.\cite{zhang2018automated}, the system is expected to run in real time, however no information regarding the
speed of the algorithm's performance is reported. Zhang et al.\cite{zhang2018automated} developed a practical near-miss trespassing detection system. Their main focus is on detection of near-miss events on gated crossings rather than general trespassing detection. They initially identify time interval during which gates are closed and then detect the moving objects using background subtraction. However, they use no. of pixels as classification metric to distinguish between train and other objects (people and vehicles). This strategy works for their test data but may fail if the camera is located away from train, since it is based on the assumption that train will always constitute most of the moving pixels. 

One of the major drawback common to all the above mentioned systems is that they simply use background subtraction for object detection. Further, the method used is mostly frame subtraction-based which is known to generate noisy output \textcolor{red}{cite}. Only Shah et al.\cite{shah2007automated} uses some hand-crafted features to classify the object type. However, recent research suggests that deep-learning based approaches outperform these hand-crafted features based approaches\cite{benenson2014ten}. This limitation in literature motivates us to develop a deep learning based trespassing detection system that can reliably detect human trespassing activity. 

