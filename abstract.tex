\begin{abstract}
%% Text of abstract
Railroad trespassing is a dangerous activity with significant security and safety risks. However, regular patrolling of potential trespassing sites is infeasible due to exceedingly high resource demands and personnel costs. This raises the need to design automated trespass detection and early warning prediction techniques leveraging state-of-the-art machine learning. To meet this need, we propose a novel framework for \underline{A}utomated \underline{R}ailroad \underline{T}respassing detection \underline{S}ystem using video surveillance data called ARTS. As the core of our solution, we adopt a CNN-based deep learning architecture capable of video processing. However, these deep learning-based methods, while effective, are known to be computationally expensive and time consuming, especially when applied to a large volume of surveillance data. Leveraging the sparsity of railroad trespassing activity, ARTS corresponds to a dual-stage deep learning architecture composed of an inexpensive pre-filtering stage for activity detection, followed by a high fidelity trespass classification stage employing deep neural network. %The former is responsible for filtering out frames that show little to no activity, this way reducing the amount of data to be processed by the later more compute-intensive stage which adopts state-of-the-art Faster-RCNN to ensure effective classification of trespassing activity. 
The resulting dual-stage ARTS architecture represents a flexible solution capable of trading-off accuracy with computational time. We demonstrate the efficacy of our approach on public domain surveillance data achieving 0.87 $f_1$ score while keeping up with the enormous video volume, achieving a practical time and accuracy trade-off. 

%taking only 50 mins to process 1 hour of survillance video in appropriate trade off. 
\end{abstract}